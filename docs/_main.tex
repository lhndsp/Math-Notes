% Options for packages loaded elsewhere
\PassOptionsToPackage{unicode}{hyperref}
\PassOptionsToPackage{hyphens}{url}
%
\documentclass[
]{book}
\usepackage{amsmath,amssymb}
\usepackage{iftex}
\ifPDFTeX
  \usepackage[T1]{fontenc}
  \usepackage[utf8]{inputenc}
  \usepackage{textcomp} % provide euro and other symbols
\else % if luatex or xetex
  \usepackage{unicode-math} % this also loads fontspec
  \defaultfontfeatures{Scale=MatchLowercase}
  \defaultfontfeatures[\rmfamily]{Ligatures=TeX,Scale=1}
\fi
\usepackage{lmodern}
\ifPDFTeX\else
  % xetex/luatex font selection
\fi
% Use upquote if available, for straight quotes in verbatim environments
\IfFileExists{upquote.sty}{\usepackage{upquote}}{}
\IfFileExists{microtype.sty}{% use microtype if available
  \usepackage[]{microtype}
  \UseMicrotypeSet[protrusion]{basicmath} % disable protrusion for tt fonts
}{}
\makeatletter
\@ifundefined{KOMAClassName}{% if non-KOMA class
  \IfFileExists{parskip.sty}{%
    \usepackage{parskip}
  }{% else
    \setlength{\parindent}{0pt}
    \setlength{\parskip}{6pt plus 2pt minus 1pt}}
}{% if KOMA class
  \KOMAoptions{parskip=half}}
\makeatother
\usepackage{xcolor}
\usepackage{color}
\usepackage{fancyvrb}
\newcommand{\VerbBar}{|}
\newcommand{\VERB}{\Verb[commandchars=\\\{\}]}
\DefineVerbatimEnvironment{Highlighting}{Verbatim}{commandchars=\\\{\}}
% Add ',fontsize=\small' for more characters per line
\usepackage{framed}
\definecolor{shadecolor}{RGB}{248,248,248}
\newenvironment{Shaded}{\begin{snugshade}}{\end{snugshade}}
\newcommand{\AlertTok}[1]{\textcolor[rgb]{0.94,0.16,0.16}{#1}}
\newcommand{\AnnotationTok}[1]{\textcolor[rgb]{0.56,0.35,0.01}{\textbf{\textit{#1}}}}
\newcommand{\AttributeTok}[1]{\textcolor[rgb]{0.13,0.29,0.53}{#1}}
\newcommand{\BaseNTok}[1]{\textcolor[rgb]{0.00,0.00,0.81}{#1}}
\newcommand{\BuiltInTok}[1]{#1}
\newcommand{\CharTok}[1]{\textcolor[rgb]{0.31,0.60,0.02}{#1}}
\newcommand{\CommentTok}[1]{\textcolor[rgb]{0.56,0.35,0.01}{\textit{#1}}}
\newcommand{\CommentVarTok}[1]{\textcolor[rgb]{0.56,0.35,0.01}{\textbf{\textit{#1}}}}
\newcommand{\ConstantTok}[1]{\textcolor[rgb]{0.56,0.35,0.01}{#1}}
\newcommand{\ControlFlowTok}[1]{\textcolor[rgb]{0.13,0.29,0.53}{\textbf{#1}}}
\newcommand{\DataTypeTok}[1]{\textcolor[rgb]{0.13,0.29,0.53}{#1}}
\newcommand{\DecValTok}[1]{\textcolor[rgb]{0.00,0.00,0.81}{#1}}
\newcommand{\DocumentationTok}[1]{\textcolor[rgb]{0.56,0.35,0.01}{\textbf{\textit{#1}}}}
\newcommand{\ErrorTok}[1]{\textcolor[rgb]{0.64,0.00,0.00}{\textbf{#1}}}
\newcommand{\ExtensionTok}[1]{#1}
\newcommand{\FloatTok}[1]{\textcolor[rgb]{0.00,0.00,0.81}{#1}}
\newcommand{\FunctionTok}[1]{\textcolor[rgb]{0.13,0.29,0.53}{\textbf{#1}}}
\newcommand{\ImportTok}[1]{#1}
\newcommand{\InformationTok}[1]{\textcolor[rgb]{0.56,0.35,0.01}{\textbf{\textit{#1}}}}
\newcommand{\KeywordTok}[1]{\textcolor[rgb]{0.13,0.29,0.53}{\textbf{#1}}}
\newcommand{\NormalTok}[1]{#1}
\newcommand{\OperatorTok}[1]{\textcolor[rgb]{0.81,0.36,0.00}{\textbf{#1}}}
\newcommand{\OtherTok}[1]{\textcolor[rgb]{0.56,0.35,0.01}{#1}}
\newcommand{\PreprocessorTok}[1]{\textcolor[rgb]{0.56,0.35,0.01}{\textit{#1}}}
\newcommand{\RegionMarkerTok}[1]{#1}
\newcommand{\SpecialCharTok}[1]{\textcolor[rgb]{0.81,0.36,0.00}{\textbf{#1}}}
\newcommand{\SpecialStringTok}[1]{\textcolor[rgb]{0.31,0.60,0.02}{#1}}
\newcommand{\StringTok}[1]{\textcolor[rgb]{0.31,0.60,0.02}{#1}}
\newcommand{\VariableTok}[1]{\textcolor[rgb]{0.00,0.00,0.00}{#1}}
\newcommand{\VerbatimStringTok}[1]{\textcolor[rgb]{0.31,0.60,0.02}{#1}}
\newcommand{\WarningTok}[1]{\textcolor[rgb]{0.56,0.35,0.01}{\textbf{\textit{#1}}}}
\usepackage{longtable,booktabs,array}
\usepackage{calc} % for calculating minipage widths
% Correct order of tables after \paragraph or \subparagraph
\usepackage{etoolbox}
\makeatletter
\patchcmd\longtable{\par}{\if@noskipsec\mbox{}\fi\par}{}{}
\makeatother
% Allow footnotes in longtable head/foot
\IfFileExists{footnotehyper.sty}{\usepackage{footnotehyper}}{\usepackage{footnote}}
\makesavenoteenv{longtable}
\usepackage{graphicx}
\makeatletter
\newsavebox\pandoc@box
\newcommand*\pandocbounded[1]{% scales image to fit in text height/width
  \sbox\pandoc@box{#1}%
  \Gscale@div\@tempa{\textheight}{\dimexpr\ht\pandoc@box+\dp\pandoc@box\relax}%
  \Gscale@div\@tempb{\linewidth}{\wd\pandoc@box}%
  \ifdim\@tempb\p@<\@tempa\p@\let\@tempa\@tempb\fi% select the smaller of both
  \ifdim\@tempa\p@<\p@\scalebox{\@tempa}{\usebox\pandoc@box}%
  \else\usebox{\pandoc@box}%
  \fi%
}
% Set default figure placement to htbp
\def\fps@figure{htbp}
\makeatother
\setlength{\emergencystretch}{3em} % prevent overfull lines
\providecommand{\tightlist}{%
  \setlength{\itemsep}{0pt}\setlength{\parskip}{0pt}}
\setcounter{secnumdepth}{5}
\usepackage{booktabs}

\usepackage[]{natbib}
\bibliographystyle{plainnat}
\usepackage{bookmark}
\IfFileExists{xurl.sty}{\usepackage{xurl}}{} % add URL line breaks if available
\urlstyle{same}
\hypersetup{
  pdftitle={Math Notes},
  pdfauthor={Lucas Porto},
  hidelinks,
  pdfcreator={LaTeX via pandoc}}

\title{Math Notes}
\author{Lucas Porto}
\date{2025-05-18}

\usepackage{amsthm}
\newtheorem{theorem}{Teorema}[chapter]
\newtheorem{lemma}{Lema}[chapter]
\newtheorem{corollary}{Corolário}[chapter]
\newtheorem{proposition}{Axioma}[chapter]
\newtheorem{conjecture}{Conjetura}[chapter]
\theoremstyle{definition}
\newtheorem{definition}{Definição}[chapter]
\theoremstyle{definition}
\newtheorem{example}{Exemple}[chapter]
\theoremstyle{definition}
\newtheorem{exercise}{Exercício}[chapter]
\theoremstyle{definition}
\newtheorem{hypothesis}{Hipótese}[chapter]
\theoremstyle{remark}
\newtheorem*{remark}{Observação }
\newtheorem*{solution}{Solução }
\begin{document}
\maketitle

{
\setcounter{tocdepth}{1}
\tableofcontents
}
\begin{Shaded}
\begin{Highlighting}[]
\NormalTok{bookdown}\SpecialCharTok{::}\FunctionTok{serve\_book}\NormalTok{()}
\end{Highlighting}
\end{Shaded}

\chapter{Definição axiomatica dos números reais}\label{definiuxe7uxe3o-axiomatica-dos-nuxfameros-reais}

\section{Axiomas e teoremas de corpo}\label{axiomas-e-teoremas-de-corpo}

\begin{proposition}[Comutatividade]
\protect\hypertarget{prp:ircomutativa}{}\label{prp:ircomutativa}\[
a + b = b + a \text{ e } ab = ba
\]
\end{proposition}

\begin{proposition}[Associatividade]
\protect\hypertarget{prp:irassociativa}{}\label{prp:irassociativa}\[
(a + b) + c = a + (c + b) \text{ e } (ab)c = a(bc)
\]
\end{proposition}

\begin{proposition}[Elemento Neutro]
\protect\hypertarget{prp:irneutro}{}\label{prp:irneutro}\[
a + 0 = a \text{ e } 1a = a
\]
\end{proposition}

\begin{proposition}[Distributividade]
\protect\hypertarget{prp:irdistributiva}{}\label{prp:irdistributiva}\[
a (b + c) = ab + cd
\]
\end{proposition}

\begin{proposition}[Simétrico ou oposto em relação a adição]
\protect\hypertarget{prp:irsimetrico}{}\label{prp:irsimetrico}\[
\forall a \in \mathbb{R} \exists! b \in \mathbb{R} \mid a + b = 0,\ b \equiv -a
\]
\end{proposition}

\begin{proposition}[Recíproco ou oposto em relação a multiplicação]
\protect\hypertarget{prp:irreciproco}{}\label{prp:irreciproco}\[
\forall a \in \mathbb{R} \exists!  b \in \mathbb{R} \mid ab = 1
\]
\end{proposition}

\begin{theorem}[Simplificação para adição]
\protect\hypertarget{thm:thmsimpladd}{}\label{thm:thmsimpladd}\[
\forall a,b,c \in \mathbb{R} \mid a+b=a+c \Rightarrow b = c
\]
\end{theorem}

\begin{theorem}[Subtração]
\protect\hypertarget{thm:thmsubtr}{}\label{thm:thmsubtr}\[
\forall a,b \in \mathbb{R} \exists! x \in \mathbb{R} \mid a + (-x) = b, \quad a-x=b 
\]
\end{theorem}

\begin{theorem}[Divisão]
\protect\hypertarget{thm:thmdiv}{}\label{thm:thmdiv}\[
\forall a,b \in \mathbb{R} \exists! x \in \mathbb{R}^* \mid xb = a, \quad \frac{a}{x}=b, \quad \frac{a}{b} = x 
\]
\end{theorem}

\begin{theorem}[Frações iguais]
\protect\hypertarget{thm:thmfrceq}{}\label{thm:thmfrceq}\[
a,c \in \mathbb{R} \wedge b,d \in \mathbb{R}^* 
\Rightarrow
\frac{a}{b} = \frac{c}{d} \Leftrightarrow ad = bc 
\]
\end{theorem}

\begin{theorem}[Recíprocos de frações]
\protect\hypertarget{thm:thmfracinv}{}\label{thm:thmfracinv}\[
a, b \in \mathbb{R}^* \Rightarrow \frac{1}{\frac{a}{b}} = \frac{b}{a}
\]
\end{theorem}

\begin{theorem}[Soma e subtração de frações]
\protect\hypertarget{thm:thmfracsum}{}\label{thm:thmfracsum}\[
a,c \in \mathbb{R} \wedge b,d \in \mathbb{R}^* 
\Rightarrow
\frac{a}{b} + \frac{c}{d} = \frac{ad + bc}{bd} 
\quad \text{e} \quad 
\frac{a}{b} - \frac{c}{d} = \frac{ad - bc}{bd}
\]
\end{theorem}

\begin{theorem}[Multiplicação de frações]
\protect\hypertarget{thm:thmfracmult}{}\label{thm:thmfracmult}\[
a \in \mathbb{R} \wedge b,c,d \in \mathbb{R}^* 
\Rightarrow
\frac{a}{b}\frac{c}{d} = \frac{ac}{bd} 
\quad \text{e} \quad
\frac{\frac{a}{b}}{\frac{c}{d}} = \frac{a}{b} \frac{d}{c}
\]
\end{theorem}

\section{Axiomas e teoremas de ordem}\label{axiomas-e-teoremas-de-ordem}

\begin{proposition}[Tricotomia]
\protect\hypertarget{prp:irtricotomia}{}\label{prp:irtricotomia}\[
a,b \in \mathbb{R} \quad  \Leftrightarrow \quad  a < b \quad \text{ ou } \quad a > b \quad \text{ ou } \quad  a = b
\]
\end{proposition}

\begin{theorem}
\protect\hypertarget{thm:thmordsum}{}\label{thm:thmordsum}\[
a, b \in \mathbb{R}^+ \Rightarrow a + b \in \mathbb{R}^+
\]
\end{theorem}

\begin{theorem}
\protect\hypertarget{thm:thmordsubrt}{}\label{thm:thmordsubrt}\[
a, b \in \mathbb{R}^* \mid a < b \Rightarrow a-b<0
\]
\end{theorem}

\begin{theorem}
\protect\hypertarget{thm:thmordmult}{}\label{thm:thmordmult}\[
\left\{\begin{matrix}
a,b \in \mathbb{R}^+ \Rightarrow ab \in \mathbb{R}^+ \\
a \in \mathbb{R}^+, b \in \mathbb{R}^- \Rightarrow ab \in \mathbb{R}^- \\
a \in \mathbb{R}^-, b \in \mathbb{R}^+ \Rightarrow ab \in \mathbb{R}^- \\
\end{matrix}\right.
\]
\end{theorem}

\begin{definition}
\protect\hypertarget{def:defordmult}{}\label{def:defordmult}\[
a,b \in \mathbb{R}^- \Rightarrow ab \in \mathbb{R}^+ 
\]
\end{definition}

\begin{theorem}
\protect\hypertarget{thm:thmorddiv}{}\label{thm:thmorddiv}\[
\forall a,b \in \mathbb{R}^* 
\left\{\begin{matrix}
|a| < |b| \Rightarrow \frac{a}{b} \in \left]-1,1 \right[\ \\
|a| > |b| \Rightarrow \frac{a}{b} \notin \left]-1,1 \right[\ \\
|a| = |b| \Rightarrow \frac{a}{b} = 1
\end{matrix}\right.
\]
\end{theorem}

\begin{theorem}
\protect\hypertarget{thm:thmordtrans}{}\label{thm:thmordtrans}\[
a, b, c \in \mathbb{R} \mid a < b \Rightarrow a+c < b+c
\]
\end{theorem}

\begin{theorem}
\protect\hypertarget{thm:thmordmulti}{}\label{thm:thmordmulti}\[
a, b \in \mathbb{R}, c \in \mathbb{R}^- \mid a < b \Rightarrow ac > bc
\]
\end{theorem}

\begin{theorem}
\protect\hypertarget{thm:thmordinv}{}\label{thm:thmordinv}\[
a, b \in \mathbb{R}^* \mid a < b \Rightarrow -a > -b
\]
\end{theorem}

\[
f(x) = x^2
\]

\chapter{Polinômios Reais}\label{polinuxf4mios-reais}

\begin{theorem}[Função polinomial]
\(P: \mathbb{R} \rightarrow \mathbb{R} \mid P(x) = a_0 + a_1x + \dots + a_nx^n = \sum_{k=1}^n a_kx^k, \forall a \in \mathbb{R}\)
\end{theorem}

Sendo assim, um polinômio é uma função definida pela soma de uma função constante e n funções potência de \(x\) sendo n o \textbf{grau} \(\delta\) do polinômio e os fatores \(a_k\) os coeficientes do polinômio, a função constante também pode ser vista como uma uma função potência de expoente 0.

\section{Operações}\label{operauxe7uxf5es}

Dados os polinômios \(f(x) = \sum_{k=1}^n a_kx^k\) e \(g(x) = \sum_{k=1}^m b_kx^k\) estão definidas as operações:

\begin{definition}[Polinômio nulo]
\protect\hypertarget{def:polinomionulo}{}\label{def:polinomionulo}\(f(x) = 0 \ \forall x \in \mathbb{R} \Leftrightarrow  \sum_{k=1}^n a_kx^k = 0 \Leftrightarrow a_k = 0\)
\end{definition}

\begin{definition}[Soma]
\protect\hypertarget{def:somapolinomio}{}\label{def:somapolinomio}\(f(x) + g(x) = \sum_{k=1}^n a_kx^k + \sum_{k=1}^m b_kx^k = \sum_{k=1}^{\max(m,n)} (a_k + b_k)x^k\)
\end{definition}

\begin{definition}[Igualdade]
\protect\hypertarget{def:igualdadepolinomio}{}\label{def:igualdadepolinomio}\(f(x) = g(x) \Leftrightarrow \sum_{k=1}^n a_kx^k = \sum_{k=1}^m b_kx^k \Leftrightarrow \sum_{k=1}^n a_kx^k - \sum_{k=1}^m b_kx^k = 0 \Leftrightarrow \sum_{k=1}^{\max(m,n)} a_kx^k - b_kx^k \Leftrightarrow \sum_{k=1}^{\max(m,n)} (a_k - b_k)x^k \Leftrightarrow a_k = b_k\)
\end{definition}

\begin{definition}[Multiplicação]
\protect\hypertarget{def:prodpolinomio}{}\label{def:prodpolinomio}\[
f(x)g(x) = 
\begin{aligned}
    &a_0b_0 
    + a_0b_1x 
    + a_0b_2x^2 
    + \cdots 
    + a_0b_mx^m \\
  +\ &a_1b_0x 
    + a_1b_1x^2 
    + a_1b_2x^3 
    + \cdots 
    + a_1b_mx^{m+1} \\
  +\ &a_2b_0x^2 
    + a_2b_1x^3 
    + a_2b_2x^4 
    + \cdots 
    + a_2b_mx^{m+2} \\
  &\vdots \\
  +\ &a_nb_0x^n 
    + a_nb_1x^{n+1} 
    + a_nb_2x^{n+2} 
    + \cdots 
    + a_nb_mx^{m+n}
\end{aligned}
\equiv \sum_{k=0}^{m+n} \left( \sum_{i=0}^k a_i b_{k-i} \right) x^k
\]
\end{definition}

\begin{definition}[Divisão]
\protect\hypertarget{def:divpolinomio}{}\label{def:divpolinomio}Na divisão \(\frac{f(x)}{g(x)}\) o objetivo é definir outros dois polinômios \(q(x)\) e \(r(x)\), chamados quociente e resto respectivamente, tais que:

\begin{enumerate}
\def\labelenumi{\arabic{enumi}.}
\tightlist
\item
  \(f(x) = g(x)q(x) + r(x)\)
\item
  \(\delta r < \delta g\)
\end{enumerate}

Por \citep{somapolinomio} temos que \(\delta f = \max (\delta qg, \delta r)\) e como \(\delta r < \delta g \Rightarrow \delta f = \delta gq = \delta g + \delta q \Rightarrow \delta q = |\delta f - \delta g|\)
\end{definition}

\chapter{Simetria de funções}\label{simetria-de-funuxe7uxf5es}

\begin{definition}[Domínio simétrico em relação à origem]
\protect\hypertarget{def:defsimetriaorigem}{}\label{def:defsimetriaorigem}Seja \(f: \mathbb{R} \rightarrow \mathbb{R}\), dizemos que \(f\) tem domínio simétrico em relação à origem se, e somente se, \(\forall x \in Df \Leftrightarrow -x \in Df\)
\end{definition}

\begin{definition}[Função par e ímpar]
\protect\hypertarget{def:defparimpar}{}\label{def:defparimpar}

Seja a função real \(f\) com domínio simétrico na origem, temos que:

\begin{itemize}
\tightlist
\item
  \(f(x)\) é \textbf{par} \(\Leftrightarrow f(x) = f(-x)\)
\item
  \(f(x)\) é \textbf{ímpar} \(\Leftrightarrow f(x) = -f(-x)\)
\end{itemize}

\end{definition}

\section{Propriedades}\label{propriedades}

Sejam \(\jmath\) e \(\Im\) classes de funções tal que \(\forall f \in \jmath\) é par e \(\forall g \in \Im\) é ímpar, temos as seguintes propriedades.

\begin{theorem}[Elemento oposto]
\(f(x) - f(-x) = 0\) e \(g(x)+g(-x)=0\)
\end{theorem}

\begin{theorem}[Somas internas à classe]
\textbf{1)} \(f_i(x) + f_j(x) \in \jmath\) e \(f_i(x) - f_j(x) \in \jmath\)

\textbf{2)} \(g_i(x) + g_j(x) \in \Im\) e \(g_i(x) - g_j(x) \in \Im\)
\end{theorem}

\begin{proof}
\[
h(x) = f_i(x)+f_j(x)=f_i(-x)+f_j(-x)=h(-x)
\]

\[
h(x) = f_i(x)-f_j(x) = f_i(-x)-f_j(-x)=h(-x)
\]

\[
h(x)=g_i(x)+g_j(x)=-g_i(-x)-g_j(-x)=-h(-x)
\]

\[
h(x)=g_i(x)-g_j(x)=g_i(x)+g_j(-x)=-h(-x)
\]
\end{proof}

\begin{theorem}[Produto interno à classe]
\(f_i(x)f_j(x) \in \jmath\) e \(g_i(x)g_j(x) \in \jmath\)
\end{theorem}

\begin{proof}
\[
f_i(x)f_j(x) = f_i(-x)f_j(-x)
\]

\[
g_i(x)g_j(x)=(-g_i(-x))(-g_j(-x))=g_i(-x)g_j(-x)
\]
\end{proof}

\begin{theorem}[Produto interclasse]
\(f(x)g(x) \in \Im\)
\end{theorem}

\begin{proof}
\[
fg(x) = f(x)g(x)=f(-x)(-g(-x))=-fg(-x)
\]
\end{proof}

\begin{theorem}[Soma interclasse]
Se \(\operatorname{Im} f \neq 0\) e \(\operatorname{Im} g \neq 0\), então seja \(h(x) = f(x) + g(x)\), temos que \(h(x) \notin \jmath\) e \(h(x) \notin \Im\)
\end{theorem}

\begin{proof}
\[
h(x) = f(x) + g(x) = f(-x) - g(-x) \therefore h(x) \neq h(-x) \wedge h(x) \neq -h(-x)
\]
\end{proof}

\begin{theorem}[Decomposição em função par e ímpar]
\(\forall h : \mathbb{R} \rightarrow \mathbb{R}\) com domínio simétrico em relação à origem, podemos definir as funções \(h_{\text{ímpar}} = h(x)-h(-x)\) e \(h_{\text{par}} = h(x)+h(-x)\) de modo que \(2h = h_{\text{par}} + h_{\text{ímpar}}\)
\end{theorem}

\begin{proof}
\textbf{Parte 1:} Mostremos que \(h_{\text{par}}(x) = \frac{h(x) + h(-x)}{2}\) é par.

\[
h_{\text{par}}(-x) = \frac{h(-x) + h(--x)}{2} = \frac{h(-x) + h(x)}{2} = h_{\text{par}}(x)
\]

\textbf{Parte 2:} Mostremos que \(h_{\text{impar}}(x) = \frac{h(x) - h(-x)}{2}\) é ímpar.

\[
h_{\text{impar}}(-x) = \frac{h(-x) - h(--x)}{2} = \frac{h(-x) - h(x)}{2} = -\frac{h(x) - h(-x)}{2} = -h_{\text{impar}}(x)
\]

\textbf{Parte 3:} Soma das partes par e ímpar:

\[
h_{\text{par}}(x) + h_{\text{impar}}(x) = \frac{h(x) + h(-x)}{2} + \frac{h(x) - h(-x)}{2} = \frac{2h(x)}{2} = h(x)
\]

Logo,

\[
h(x) = \frac{h(x) + h(-x)}{2} + \frac{h(x) - h(-x)}{2}
\]
\end{proof}

  \bibliography{book.bib,packages.bib}

\end{document}
