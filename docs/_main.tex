% Options for packages loaded elsewhere
\PassOptionsToPackage{unicode}{hyperref}
\PassOptionsToPackage{hyphens}{url}
%
\documentclass[
]{book}
\usepackage{amsmath,amssymb}
\usepackage{iftex}
\ifPDFTeX
  \usepackage[T1]{fontenc}
  \usepackage[utf8]{inputenc}
  \usepackage{textcomp} % provide euro and other symbols
\else % if luatex or xetex
  \usepackage{unicode-math} % this also loads fontspec
  \defaultfontfeatures{Scale=MatchLowercase}
  \defaultfontfeatures[\rmfamily]{Ligatures=TeX,Scale=1}
\fi
\usepackage{lmodern}
\ifPDFTeX\else
  % xetex/luatex font selection
\fi
% Use upquote if available, for straight quotes in verbatim environments
\IfFileExists{upquote.sty}{\usepackage{upquote}}{}
\IfFileExists{microtype.sty}{% use microtype if available
  \usepackage[]{microtype}
  \UseMicrotypeSet[protrusion]{basicmath} % disable protrusion for tt fonts
}{}
\makeatletter
\@ifundefined{KOMAClassName}{% if non-KOMA class
  \IfFileExists{parskip.sty}{%
    \usepackage{parskip}
  }{% else
    \setlength{\parindent}{0pt}
    \setlength{\parskip}{6pt plus 2pt minus 1pt}}
}{% if KOMA class
  \KOMAoptions{parskip=half}}
\makeatother
\usepackage{xcolor}
\usepackage{color}
\usepackage{fancyvrb}
\newcommand{\VerbBar}{|}
\newcommand{\VERB}{\Verb[commandchars=\\\{\}]}
\DefineVerbatimEnvironment{Highlighting}{Verbatim}{commandchars=\\\{\}}
% Add ',fontsize=\small' for more characters per line
\usepackage{framed}
\definecolor{shadecolor}{RGB}{248,248,248}
\newenvironment{Shaded}{\begin{snugshade}}{\end{snugshade}}
\newcommand{\AlertTok}[1]{\textcolor[rgb]{0.94,0.16,0.16}{#1}}
\newcommand{\AnnotationTok}[1]{\textcolor[rgb]{0.56,0.35,0.01}{\textbf{\textit{#1}}}}
\newcommand{\AttributeTok}[1]{\textcolor[rgb]{0.13,0.29,0.53}{#1}}
\newcommand{\BaseNTok}[1]{\textcolor[rgb]{0.00,0.00,0.81}{#1}}
\newcommand{\BuiltInTok}[1]{#1}
\newcommand{\CharTok}[1]{\textcolor[rgb]{0.31,0.60,0.02}{#1}}
\newcommand{\CommentTok}[1]{\textcolor[rgb]{0.56,0.35,0.01}{\textit{#1}}}
\newcommand{\CommentVarTok}[1]{\textcolor[rgb]{0.56,0.35,0.01}{\textbf{\textit{#1}}}}
\newcommand{\ConstantTok}[1]{\textcolor[rgb]{0.56,0.35,0.01}{#1}}
\newcommand{\ControlFlowTok}[1]{\textcolor[rgb]{0.13,0.29,0.53}{\textbf{#1}}}
\newcommand{\DataTypeTok}[1]{\textcolor[rgb]{0.13,0.29,0.53}{#1}}
\newcommand{\DecValTok}[1]{\textcolor[rgb]{0.00,0.00,0.81}{#1}}
\newcommand{\DocumentationTok}[1]{\textcolor[rgb]{0.56,0.35,0.01}{\textbf{\textit{#1}}}}
\newcommand{\ErrorTok}[1]{\textcolor[rgb]{0.64,0.00,0.00}{\textbf{#1}}}
\newcommand{\ExtensionTok}[1]{#1}
\newcommand{\FloatTok}[1]{\textcolor[rgb]{0.00,0.00,0.81}{#1}}
\newcommand{\FunctionTok}[1]{\textcolor[rgb]{0.13,0.29,0.53}{\textbf{#1}}}
\newcommand{\ImportTok}[1]{#1}
\newcommand{\InformationTok}[1]{\textcolor[rgb]{0.56,0.35,0.01}{\textbf{\textit{#1}}}}
\newcommand{\KeywordTok}[1]{\textcolor[rgb]{0.13,0.29,0.53}{\textbf{#1}}}
\newcommand{\NormalTok}[1]{#1}
\newcommand{\OperatorTok}[1]{\textcolor[rgb]{0.81,0.36,0.00}{\textbf{#1}}}
\newcommand{\OtherTok}[1]{\textcolor[rgb]{0.56,0.35,0.01}{#1}}
\newcommand{\PreprocessorTok}[1]{\textcolor[rgb]{0.56,0.35,0.01}{\textit{#1}}}
\newcommand{\RegionMarkerTok}[1]{#1}
\newcommand{\SpecialCharTok}[1]{\textcolor[rgb]{0.81,0.36,0.00}{\textbf{#1}}}
\newcommand{\SpecialStringTok}[1]{\textcolor[rgb]{0.31,0.60,0.02}{#1}}
\newcommand{\StringTok}[1]{\textcolor[rgb]{0.31,0.60,0.02}{#1}}
\newcommand{\VariableTok}[1]{\textcolor[rgb]{0.00,0.00,0.00}{#1}}
\newcommand{\VerbatimStringTok}[1]{\textcolor[rgb]{0.31,0.60,0.02}{#1}}
\newcommand{\WarningTok}[1]{\textcolor[rgb]{0.56,0.35,0.01}{\textbf{\textit{#1}}}}
\usepackage{longtable,booktabs,array}
\usepackage{calc} % for calculating minipage widths
% Correct order of tables after \paragraph or \subparagraph
\usepackage{etoolbox}
\makeatletter
\patchcmd\longtable{\par}{\if@noskipsec\mbox{}\fi\par}{}{}
\makeatother
% Allow footnotes in longtable head/foot
\IfFileExists{footnotehyper.sty}{\usepackage{footnotehyper}}{\usepackage{footnote}}
\makesavenoteenv{longtable}
\usepackage{graphicx}
\makeatletter
\newsavebox\pandoc@box
\newcommand*\pandocbounded[1]{% scales image to fit in text height/width
  \sbox\pandoc@box{#1}%
  \Gscale@div\@tempa{\textheight}{\dimexpr\ht\pandoc@box+\dp\pandoc@box\relax}%
  \Gscale@div\@tempb{\linewidth}{\wd\pandoc@box}%
  \ifdim\@tempb\p@<\@tempa\p@\let\@tempa\@tempb\fi% select the smaller of both
  \ifdim\@tempa\p@<\p@\scalebox{\@tempa}{\usebox\pandoc@box}%
  \else\usebox{\pandoc@box}%
  \fi%
}
% Set default figure placement to htbp
\def\fps@figure{htbp}
\makeatother
\setlength{\emergencystretch}{3em} % prevent overfull lines
\providecommand{\tightlist}{%
  \setlength{\itemsep}{0pt}\setlength{\parskip}{0pt}}
\setcounter{secnumdepth}{5}
\usepackage{booktabs}
\usepackage[]{natbib}
\bibliographystyle{plainnat}
\usepackage{bookmark}
\IfFileExists{xurl.sty}{\usepackage{xurl}}{} % add URL line breaks if available
\urlstyle{same}
\hypersetup{
  pdftitle={Math Notes},
  pdfauthor={Lucas Porto},
  hidelinks,
  pdfcreator={LaTeX via pandoc}}

\title{Math Notes}
\author{Lucas Porto}
\date{2025-05-16}

\usepackage{amsthm}
\newtheorem{theorem}{Teorema}[chapter]
\newtheorem{lemma}{Lema}[chapter]
\newtheorem{corollary}{Corolário}[chapter]
\newtheorem{proposition}{Proposição}[chapter]
\newtheorem{conjecture}{Conjetura}[chapter]
\theoremstyle{definition}
\newtheorem{definition}{Definição}[chapter]
\theoremstyle{definition}
\newtheorem{example}{Exemple}[chapter]
\theoremstyle{definition}
\newtheorem{exercise}{Exercício}[chapter]
\theoremstyle{definition}
\newtheorem{hypothesis}{Hipótese}[chapter]
\theoremstyle{remark}
\newtheorem*{remark}{Observação }
\newtheorem*{solution}{Solução }
\begin{document}
\maketitle

{
\setcounter{tocdepth}{1}
\tableofcontents
}
\begin{Shaded}
\begin{Highlighting}[]
\NormalTok{bookdown}\SpecialCharTok{::}\FunctionTok{serve\_book}\NormalTok{()}
\end{Highlighting}
\end{Shaded}

\chapter{Números reais}\label{nuxfameros-reais}

\section{Axiomas e teoremas}\label{axiomas-e-teoremas}

\citep{anton2000}

\subsection{Corpo}\label{corpo}

Defini-se como \textbf{corpo} qualquer conjunto \(\mathbb{K}\) no qual estão definidas as operações de soma e multiplicação que satisfazem as propriedades:

\begin{equation} 
\forall x \in \mathbb{R} ~~ \exists! ~~  a,b,c,d \in \mathbb{R} \mid x = a + b, x = cd\label{eq:ra0}
\end{equation} \begin{equation} 
(a + b) + c = a + (c + b) \text{ e } (ab)c = a(bc)\label{eq:ra1}
\end{equation} \begin{equation} 
a + b = b + a \text{ e } ab = ba\label{eq:ra2}
\end{equation} \begin{equation} 
a + 0 = a \text{ e } 1a = a\label{eq:ra3}
\end{equation} \begin{equation} 
a (b + c) = ab + cd\label{eq:ra4}
\end{equation} \begin{equation} 
\forall a \in \mathbb{R} \exists! b \in \mathbb{R} \mid a + b = 0\label{eq:ra5}
\end{equation} \begin{equation} 
\forall a \in \mathbb{R} \exists!  b \in \mathbb{R} \mid ab = 1\label{eq:ra6}
\end{equation} \begin{equation} 
-(-a) = a\label{eq:ra7}
\end{equation}

São exemplos de corpos os conjuntos \(\mathbb{Q}\), \(\mathbb{R}\) e \(\mathbb{C}\), sendo que \(\mathbb{N}\) não considerado um corpo por não ter definido o elemento neutro multiplicativo e aditivivo e \(\mathbb{Z}\) não é um corpo por não possuir recíproco.

Das propriedades definidas acima, pode-se deduzir os teoremas:

\begin{theorem}[Simplificação para adição]
\protect\hypertarget{thm:thmsimpladd}{}\label{thm:thmsimpladd}\[
\forall a,b,c \in \mathbb{R} \mid a+b=a+c \Rightarrow b = c
\]
\end{theorem}

\begin{proof}
Por \eqref{eq:ra5} temos que \(\exists! y \in \mathbb{R} \mid a+y = 0 \Rightarrow y+a+b=y+a+c\) e por \eqref{eq:ra2} \(0+b = 0+c \Rightarrow b = c\)
\end{proof}

\begin{theorem}[Subtração]
\protect\hypertarget{thm:thmsubtr}{}\label{thm:thmsubtr}\[
\forall a,b \in \mathbb{R} \exists! x \in \mathbb{R} \mid a + (-x) = b, \quad a-x=b 
\]
\end{theorem}

\begin{proof}
Sabe-se que \(\exists!y \mid a+y = 0\), esse \(y\) é chamado \(-a\) logo \(a+(-x)+(-a)=b+(-a) \Leftrightarrow -x = b + (-a)\), então podemos reescrever a equação como \(a + b + (-a) = a + b + (-a) \equiv b = b\) que é verdadeira.
\end{proof}

\begin{theorem}[Divisão]
\protect\hypertarget{thm:thmdiv}{}\label{thm:thmdiv}\[
\forall a,b \in \mathbb{R} \exists! x \in \mathbb{R}^* \mid xb = a, \quad \frac{a}{x}=b, \quad \frac{a}{b} = x 
\]
\end{theorem}

\begin{proof}
\(\exists!y\mid by=1 \Rightarrow xby=ay \Rightarrow x = ay \Rightarrow ayb = ayb\) Sendo que essa ultima igualdade resulta em uma expressão verdadeira
\end{proof}

\begin{theorem}[Frações iguais]
\protect\hypertarget{thm:thmfrceq}{}\label{thm:thmfrceq}\[
a,c \in \mathbb{R} \wedge b,d \in \mathbb{R}^* 
\Rightarrow
\frac{a}{b} = \frac{c}{d} \Leftrightarrow ad = bc 
\]
\end{theorem}

\begin{proof}
Temos que \(\frac{a}{b}=\frac{c}{d} \Leftrightarrow \frac{a}{b} - \frac{c}{d} = 0\) e por \ref{thm:thmfracsum} \(\frac{a}{b} - \frac{c}{d} = \frac{ab - bc}{bd} = 0 \Leftrightarrow ab - bc = 0 \equiv ad = bc\)
\end{proof}

\begin{theorem}[Recíprocos de frações]
\protect\hypertarget{thm:thmfracinv}{}\label{thm:thmfracinv}\[
a, b \in \mathbb{R}^* \Rightarrow \frac{1}{\frac{a}{b}} = \frac{b}{a}
\]
\end{theorem}

\begin{proof}[Demonstração]
Por \eqref{eq:ra6} \(\exists!x \in \mathbb{R}^* \mid \frac{a}{b}x = 1 \Leftrightarrow ax = b\) Logo \(x = \frac{b}{a}\)
\end{proof}

\begin{theorem}[Soma e subtração de frações]
\protect\hypertarget{thm:thmfracsum}{}\label{thm:thmfracsum}\[
a,c \in \mathbb{R} \wedge b,d \in \mathbb{R}^* 
\Rightarrow
\frac{a}{b} + \frac{c}{d} = \frac{ad + bc}{bd} 
\quad \text{e} \quad 
\frac{a}{b} - \frac{c}{d} = \frac{ad - bc}{bd}
\]
\end{theorem}

\begin{proof}
\$\$

\$\$
\end{proof}

\begin{theorem}[Multiplicação de frações]
\protect\hypertarget{thm:thmfracmult}{}\label{thm:thmfracmult}\[
a \in \mathbb{R} \wedge b,c,d \in \mathbb{R}^* 
\Rightarrow
\frac{a}{b}\frac{c}{d} = \frac{ac}{bd} 
\quad \text{e} \quad
\frac{\frac{a}{b}}{\frac{c}{d}} = \frac{a}{b} \frac{d}{c}
\]
\end{theorem}

\begin{proof}
Temos por \eqref{eq:ra6} que \(\frac{a}{b}\frac{c}{d} = a\frac{1}{b}c\frac{1}{d}\) e por \eqref{eq:ra2} \(a\frac{1}{b}c\frac{1}{d} = ac\frac{1}{b}\frac{1}{d} = \frac{ac}{db}\). Ja para \(\frac{\frac{a}{b}}{\frac{c}{d}} = \frac{a}{b}\frac{1}{\frac{c}{d}} = \frac{a}{b} \frac{d}{c}\), por \ref{thm:thmfracinv}
\end{proof}

\subsection{Ordem}\label{ordem}

Um \textbf{corpo} \(K\) é dito ordenado se apartir de seu eixo de simetria, chamado \(0\), e seus subconjuntos \(K^+|\forall k \in K \Rightarrow a > 0\) e \(K^-|\forall k \in K \Rightarrow a < 0\), quaisquer dos elementos \(a, b, c \in K\), satisfazem as propriedades:

\begin{equation}
a,b \in \mathbb{R} \Leftrightarrow \left\{\begin{matrix} a < b\\ a > b \\ a = b \end{matrix}\right. \label{eq:tricotomia}
\end{equation}

\begin{theorem}
\protect\hypertarget{thm:thmordsum}{}\label{thm:thmordsum}\[
a, b \in \mathbb{R}^+ \Rightarrow a + b \in \mathbb{R}^+
\]
\end{theorem}

\begin{theorem}
\protect\hypertarget{thm:thmordsubrt}{}\label{thm:thmordsubrt}\[
a, b \in \mathbb{R}^* \mid a < b \Rightarrow a-b<0
\]
\end{theorem}

\begin{theorem}
\protect\hypertarget{thm:thmordmult}{}\label{thm:thmordmult}\[
\left\{\begin{matrix}
a,b \in \mathbb{R}^+ \Rightarrow ab \in \mathbb{R}^+ \\
a \in \mathbb{R}^+, b \in \mathbb{R}^- \Rightarrow ab \in \mathbb{R}^- \\
a \in \mathbb{R}^-, b \in \mathbb{R}^+ \Rightarrow ab \in \mathbb{R}^- \\
\end{matrix}\right.
\]
\end{theorem}

\begin{definition}
\protect\hypertarget{def:defordmult}{}\label{def:defordmult}\[
a,b \in \mathbb{R}^- \Rightarrow ab \in \mathbb{R}^+ 
\]
\end{definition}

\begin{theorem}
\protect\hypertarget{thm:thmorddiv}{}\label{thm:thmorddiv}\[
\forall a,b \in \mathbb{R}^* 
\left\{\begin{matrix}
|a| < |b| \Rightarrow \frac{a}{b} \in \left]-1,1 \right[\ \\
|a| > |b| \Rightarrow \frac{a}{b} \notin \left]-1,1 \right[\ \\
|a| = |b| \Rightarrow \frac{a}{b} = 1
\end{matrix}\right.
\]
\end{theorem}

\begin{theorem}
\protect\hypertarget{thm:thmordtrans}{}\label{thm:thmordtrans}\[
a, b, c \in \mathbb{R} \mid a < b \Rightarrow a+c < b+c
\]
\end{theorem}

\begin{theorem}
\protect\hypertarget{thm:thmordmulti}{}\label{thm:thmordmulti}\[
a, b \in \mathbb{R}, c \in \mathbb{R}^- \mid a < b \Rightarrow ac > bc
\]
\end{theorem}

\begin{theorem}
\protect\hypertarget{thm:thmordinv}{}\label{thm:thmordinv}\[
a, b \in \mathbb{R}^* \mid a < b \Rightarrow -a > -b
\]
\end{theorem}

  \bibliography{book.bib,packages.bib}

\end{document}
